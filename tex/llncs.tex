% This is LLNCS.DOC the documentation file of
% the LaTeX2e class from Springer-Verlag
% for Lecture Notes in Computer Science, version 2.4
\documentclass{llncs}
\usepackage{llncsdoc}
\usepackage{tikz}
\usepackage{array}
\usepackage{multicol}
\usepackage{mathtools}
\usepackage{float}
\usepackage{caption}
\usepackage{rotating}
\usepackage{longtable}
\usepackage{listings}
\usepackage[hyphens]{url}

%
\begin{document}


	{
	%
	\title{title}
	
	\author{Benjamin Leiding 1\inst{1} \and Will Vorobev\inst{1} \and Peter Zverkov\inst{1} \and Lena Cherry\inst{1}}
	
	\institute{ 
		Chorus Technology
		}
	
	%\institute{University of G\"ottingen, Institute of Computer Science, G\"ottingen, Germany\\ benjamin.leiding@cs.uni-goettingen.de
	%\and
	%Tallinn University of Technology, Department of Informatics, Tallinn, Estonia\\
	%alex.norta.phd@ieee.org}
	
	\maketitle

	%% ----------------------------------------------------------------
	%% ----------------------------------------------------------------

	\begin{abstract}
		
		I am an abstract - pet me.

		
	\end{abstract}
	
	
	\keywords{keywords}

	%% ----------------------------------------------------------------
	%% ----------------------------------------------------------------
	
	\section{Introduction}
		\label{s:introduction}
	

		The remainder of this whitepaper is structured as follows: 
%		Section \ref{section-2} focuses on the difficulties and challenges posed by ESAs as well as the role and potential of information technology in such activities. Section \ref{section-3} deals with example use cases and scenarios used to derive technical requirements and functionalities of emergency devices for ESAs. The technical requirements are discussed in Section \ref{section-4}. Section \ref{section-5} introduces a prototype implementation of a emergency fall detection device for climbers, followed by the evaluation of the prototype in Section \ref{section-6} and a discussion on potential enhancement of the prototype for broader applications. Section \ref{section-7} concludes this whitepaper together with discussing limitations, open issues and future development work.


	%% ----------------------------------------------------------------
	%% ----------------------------------------------------------------

	\section{Section 2}	
		\label{section-2}

	

	%% ----------------------------------------------------------------
	%% ----------------------------------------------------------------

	\section{Section 3}
		\label{section-3}
		
	
	%% ----------------------------------------------------------------
	%% ----------------------------------------------------------------
	
	
	\section{Section 4}
		\label{section-4}	
		
	

	%% ----------------------------------------------------------------
	%% ----------------------------------------------------------------

	\section{Section 5}
		\label{section-5}	

	%% ----------------------------------------------------------------
	%% ----------------------------------------------------------------

	\section{Section 5}
		\label{section-6}	
	
	%% ----------------------------------------------------------------
	%% ----------------------------------------------------------------

	\section{Conclusion}
		\label{section-7}	

				
	%% ----------------------------------------------------------------
	%% ----------------------------------------------------------------


	\label{Bibliography}
	\bibliographystyle{splncs03}
	\bibliography{Bibliography}
	
	%% ----------------------------------------------------------------
	%% ----------------------------------------------------------------	



	%
\end{document}
